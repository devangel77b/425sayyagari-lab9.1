\documentclass[conference]{IEEEtran}
\IEEEoverridecommandlockouts
% The preceding line is only needed to identify funding in the first footnote. If that is unneeded, please comment it out.
%Template version as of 6/27/2024

\usepackage{cite}
\usepackage{amsfonts}
\usepackage{amsmath}
\usepackage{physics}
\usepackage{amssymb}
\usepackage{algorithmic}
\usepackage{graphicx}
\usepackage{textcomp}
\usepackage{xcolor}
\def\BibTeX{{\rm B\kern-.05em{\sc i\kern-.025em b}\kern-.08em
    T\kern-.1667em\lower.7ex\hbox{E}\kern-.125emX}}
\begin{document}

\title{Computational Mapping Analysis of Equipotential and Electric Field Lines in Gel Electrophoresis Rig}\\

\author{\IEEEauthorblockN{Saketh Ayyagari}
\IEEEauthorblockA{\textit{SE Magnet Program} \\
\textit{Manalapan HS}\\
Manalapan, US \\
425sayyagari@frhsd.com}
\and
\IEEEauthorblockN{Victoria Collemi}
\IEEEauthorblockA{\textit{SE Magnet Program} \\
\textit{Manalapan HS}\\
Manalapan, US \\
425vcollemi@frhsd.com}
\and
\IEEEauthorblockN{Krish Shah}
\IEEEauthorblockA{\textit{SE Magnet Program} \\
\textit{Manalapan HS}\\
Manalapan, US \\
425kshah@frhsd.com}
\and
\IEEEauthorblockN{Kevin Tomazic}
\IEEEauthorblockA{\textit{SE Magnet Program} \\
\textit{Manalapan HS}\\
Manalapan, US \\
425ktomazic@frhsd.com}

}

\maketitle

\begin{abstract}
Equipotential curves, also known as isolines, are lines of constant electric potential where a charged object does no work moving along them. These lines determine the direction of an electric field (represented by a vector field) as the direction of the field at a point will be perpendicular to an equipotential line. To simulate these equipotential lines, we used two oppositely charged hex nuts to observe the relationship between potential and electric fields. After measuring the potential at every point on a 6 x 6 plane, we used Matplotlib to generate graphs of both equipotentials and predicted electric field lines. These experimental findings can help predict the path DNA molecules will take through a gel electrophoresis rig.
\end{abstract}

\maketitle
\IEEEdisplaynontitleabstractindextext


\begin{IEEEkeywords}
electric potential, equipotential lines, electric field visualizations, saline conductive medium, electrostatics, electric field mapping, data visualization in electrostatics, gel electrophoresis, electric field dynamics.
\end{IEEEkeywords}

\section{Introduction}
The electric potential V of a particle can be described as the amount of electrical potential energy per unit coulomb of charge. This can be expressed by the following equation:

\begin{center}
$V = \frac{U_E}{q} = k \frac{Q}{r}$
\end{center}

\noindent where $U_E$ is the electric potential energy between a charge Q and a small test charge q, and r is the distance from the point of interest to the location of the charge Q. Multiple points along a plane where the potential V is equal can be represented by equipotential lines. Using these equipotential lines, the electric field $\vec{E}$ generated by the charge Q can be described as:

% \begin{equation}
% E_x = -\frac{dV}{dx}, E_y = -\frac{dV}{dy}\label{eq}
% \end{equation}

% \noindent where X and Y describe directions perpendicular to each other. This also means the electric field $\hat{E}$ can be be described as: 
\begin{equation}
\vec{E} = -\vec{\nabla}{V}    
\end{equation}

\noindent where $\vec{\nabla}$ is the gradient of a multivariable function. These equations indicate that a larger electric potential difference with respect to a particular direction results in a larger magnitude of the electric field antiparallel to that direction. This implies that the electric field will always point in the direction of decreasing electric potential and be perpendicular to the isolines. In this experiment, we analyze the equipotential regions and their relationship associated with the electric fields generated by different electrode configurations in a saline medium. Such analyses have practical findings when designing gel electrophoresis rigs as an electric field can predict the path a negatively charged DNA molecule will take when moving from the negatively charged region to a positively charged one \cite{b1}. 

\section{Methods and Materials}
To measure the voltage, we used a Rubbermaid 3-cup container, laminated graphing paper, a Buck Engineering Lab-Volt 187 power supply, 2 M10 metal hex nuts, a DM 1800 digital multimeter, 4 leads, and a smartphone camera. We filled our container with water to approximately 0.01m in depth and added a pinch of salt. Then, we set the power supply voltage to around 10 volts (V); however, the true output voltage was 8.89 V. We then attached 2 leads to connect the power supply to the hex nuts, which were placed on opposite sides of the container. On the graph paper. We also marked 36 equally spaced points in a 6x6 square to measure the voltage, which was placed under the container so the center of the square was under the center of the container. Using our smartphone, we took pictures of each measurement. After the first trial, we rotated the orientation of the hex nuts 90 degrees clockwise, with the positive lead at the top of the container and the negative lead at the bottom. We repeated the same measurement process from the first trial in the second.

After collecting our data, we utilized the Numpy and Matplotlib libraries to generate visuals of the equipotential and electric field lines. To generate coordinate pairs for our system, we used Numpy’s $\texttt{meshgrid()}$ method \cite{b2}, which takes in lists of x and y coordinates where the voltage was measured. For equipotential lines, we used Matplotlib’s $\texttt{contour()}$ method \cite{b3, b4}, which takes in two matrices generated by the $\texttt{meshgrid()}$  function along with the level at each point. We also used the $\texttt{colorbar()}$ method to show what color line represents which quantity of potential. To generate the electric field, we first approximate the partial derivatives of the potential with respect to the x and y directions, resulting in a vector at each point perpendicular to the level curve. We then use Matplotlib’s $\texttt{quiver()}$ method \cite{b5}, which takes the outputs of the $\texttt{meshgrid()}$  function along with matrices containing the components of the vector at each coordinate to generate a vector field. 

\begin{figure}
\centerline{\includegraphics[scale=0.2]{SE Physics E&M - Lab Report.png}}
\caption{Diagram of our setup}
\end{figure}

\section{Results}

\subsection{Positive Nut (Left), Negative Nut (Right)}
\begin{figure}
\centerline{\includegraphics[scale=0.64]{trial1.png}} % Positive nut on left, Negative nut on right
\caption{Equipotential lines and electric field in the first configuration. The red and blue hexagons represent the positive and negative nuts, respectively}
\end{figure}

For the first trial, the negatively charged nut was placed on the left and the positively charged nut on the right. Our power supply applied a measured voltage of 8.89 V to our experimental system.
In Fig. 2, we observed the equipotential lines of larger magnitude were closer to the positively charged nut and those of lower potential were closer to the negatively charged nut. Between the y-coordinate values of 3 and 4, we calculated the experimental value of the equipotential line approximately equidistant to the two oppositely charged nuts to be 4.65 V $\frac{4.8+4.5}{2}$. In addition, as the position moved from the location of the positively charged nut to that of the negatively charged nut, the line density of the equipotentials decreased. To obtain an average rate of change of the electric potential in the direction of decreasing potential for this trial, we used the following equation:

\begin{equation}
-\frac{\Delta V}{\Delta s} = -\frac{V_{max neg} - V_{max pos}}{d}\label{eq}
\end{equation}

\noindent where $\Delta V$ is the potential difference between the maximum electric potential near the negatively charged nut and the maximum electric potential near the positively charged nut, and $d$ is the straight line distance between the two nuts. Referencing Fig. 2 for the electric potential values, $V_{max neg} = 3.3 V$ and $V_{max pos} = 6.9 V$. To determine the value of $d$, we can treat the charged nuts as point charges, so $d$ is the straight-line distance between them since there were six points of measurement separated by approximately 0.01 m (or ½ inch): $d= 5 * 0.01 = 0.05$ m. Therefore, we can substitute the experimental values for $V_{max neg}$, $V_{max pos}$, and $d$ to find that the average rate of change of the electric potential in the direction of decreasing potential is 72 V/m.

\subsection{Positive Nut (Top), Negative Nut (Bottom)}
\begin{figure}
\centerline{\includegraphics[scale=0.64]{trial2.png}} % Negative nut going in the South direction and Positive nut going in North direction
\caption{Equipotential lines and electric field in the second configuration. The red and blue hexagons represent the positive and negative nuts, respectively}
\end{figure}
For the second trial, the positively charged nut was placed on the top (North) and the negatively charged nut was placed on the bottom (South). Our power supply applied a measured voltage of 8.46 V to our experimental system.

In Fig. 3, we observed that the equipotential lines of higher electric potential were closer to the positively charged nut and those of lower electric potential were closer to the negatively charged nut. Between the x-coordinate (Fig. 3 is supposed to be rotated 90 degrees clockwise) values of 3 and 4, we calculated the experimental value of the equipotential line approximately equidistant to the two oppositely charged nuts to be 5 V $\frac{5.2+4.8}{2}$. In addition, as the position moved from the location of the positively charged nut to that of the negatively charged nut, the line density of the equipotentials decreased. 

To obtain an average rate of change of the electric potential in the direction of decreasing potential for this trial, we can use Eq 2 from Trial 1. Referencing Fig. 3 for the electric potential values, $V_{maxneg} = 3.2 V$ and $V_{maxpos} = 6.8 V$. Since the value of $d$ is the same for both trials, we can substitute the experimental values for $V_{maxneg}$, $V_{maxpos}$, and $d$ to find that the average rate of change of the electric potential in the direction of decreasing potential is 72 V/m. This value is the same rate of change of the electric potential as the first trial, demonstrating consistency across both a change in charge orientation (rotating the nuts’ locations by 90 degrees) and the potential difference.


\section{Discussion}
Based on the applied voltages from Trials 1 and 2, the theoretical value for the equipotential line equidistant to the two oppositely charged hex nuts is 4.45 V and 4.23 V, respectively, since they are expected to be exactly half of the total voltage supplied to each system (Trial 1: 8.89 V; Trial 2: 8.46 V). To determine the percent error of our observed voltage values, we can use the percent error equation:

\begin{equation}
\delta = \frac{v_a - v_e}{v_e} \times 100
\end{equation}

Where $\delta$ is the percent error, $v_a$ is the experimental voltage, and $v_e$ is the expected voltage. Substituting in the proper values, we found that $\delta_1$ = 4.49\%, indicating that our experimental voltage was close to the expected value for Trial 1. In contrast, we found that $\delta_2$ = 18.20\%, indicating that our experimental voltage was not close to the expected value for Trial 2.

According to theory, the electric field points in the direction of decreasing electric potential (Eq. 1) and is perpendicular to the equipotential lines. %since no work is done when moving along an isoline%. 
To be consistent with the equipotentials in Fig. 2 and Fig. 3, the electric field must point in the direction of decreasing electric potential. In addition, the magnitude of the electric field at a point in space should be higher in regions where the equipotential lines are more densely packed together (near the positively charged nut) and lower in regions where they are more spaced out (near the negatively charged nut).

In Figs. 2 and 3, as the position moves down the gradient of electric potential, the density of the isolines decreases. This is consistent with the direction of the experimental average rates of change in electric potential found for Figs. 2 and 3 (-72 V/m for both). This implies the presence of an electric field in both trials since the overall vector field direction is consistent with the behavior predicted by Eq. 1. In addition, the regions in Figs. 2 and 3 where the magnitude of the electric field vectors (represented by the vector length) is at a maximum is consistent with the isoline density distribution in both equipotential maps, where areas of higher equipotential line density correspond to regions of high electric field strength. The observed consistency of the electric field vectors and isoline orientation and densities in Figs. 2 and 3 indicate the validity of using our measuring setup for a device governed by electrostatic principles. Our measurement techniques could be applied to devices similar to the electrophoresis rig we tested, including capacitor designs and electric field sensors. 

Any outlying values of all experimentally determined quantities and the discrete nature of our equipotential lines across experiments are most likely due to the limited number of points measured and measurement errors (e.g. not measuring at the exact point, oscillating voltage values on the voltmeter). Across both experiments, data was collected at only 36 points for each trial, a very small dataset. This may have contributed to the straight, discrete nature of the equipotential lines since there was not enough data to generate curved, smoother lines. In addition, more configurations of the two-nut system should be tested to observe how the isolines and voltage values change with respect to the orientation of the system.

\section{Conclusion}
This experiment illustrated the relationship between theoretical knowledge of electric fields and their practical visualization through equipotential lines in salt water. By mapping these regions, we validated the fundamental relationship between electric potential and electric fields as described by electrostatic theory. The consistency of electric field vectors and equipotential line directions corroborates that our experimental setup can simulate electrostatic phenomena in other engineering applications, including capacitors, electric field sensors, and devices like electrophoresis setups where accurate electric field mapping is essential. Next time, we could also try different electrode configurations to visualize different cases and collect more data points to get smoother equipotential curves. In future experiments, we could develop electrostatic simulations to observe the effects of measurement on the electric potential in the electrophoresis rig. Through the fusion of theory and application, this work provides a solid ground to apply electrostatics to new engineering solutions.  

\section*{Contributions}
Saketh Ayyagari contributed to the abstract, methods description, and data visualization components. Victoria Collemi contributed to the introduction, results, discussion, and paper formatting. Krish Shah and Kevin Tomazic were responsible for data collection and the experimental setup. Kevin Tomazic also contributed to the methods section along with creating the setup diagram. 

\begin{thebibliography}{00}
\bibitem{b1} “LabXchange,” www.labxchange.org. https://www.labxchange.org/library/items/lb:LabXchange:bee8a688:html:1
\bibitem{b2} "numpy.meshgrid: NumPy v1.22 Manual," numpy.org. https://numpy.org/doc/stable/reference/generated/numpy.meshgrid.html
\bibitem{b3} J. T. Vanderplas, “Density and Contour Plots,” in Python Data Science Handbook: Essential Tools for Working with Data, Beijing Etc.: O’Reilly, Cop, 2016.
\bibitem{b4} "matplotlib.axes.Axes.contour: Matplotlib 3.9.2 documentation," Matplotlib.org, 2014. https://matplotlib.org/stable/api/\_as\_gen/matplotlib.axes.Axes.contour.html (accessed Nov. 15, 2024).
\bibitem{b5} "matplotlib.pyplot.quiver: Matplotlib 3.6.2 documentation," matplotlib.org. https://matplotlib.org/stable/api/\_as\_gen/matplotlib.pyplot.quiver.html
\bibitem{b6} Saketh-Ayyagari, “GitHub - Saketh-Ayyagari/Mapping-Equipotential-Lines-and-Electric-Fields: Code used to generate equipotential lines and electric field for an S\&E AP Physics C: E\&M Lab.,” GitHub, 2024. https://github.com/Saketh-Ayyagari/Mapping-Equipotential-Lines-and-Electric-Fields (accessed Dec. 02, 2024).
‌
‌

\end{thebibliography}

\end{document}

